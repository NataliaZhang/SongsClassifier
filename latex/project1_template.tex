\documentclass{article}
\newif\ifshowsolutions
\showsolutionstrue
\input{./preamble}

%%%%%%%%%%%%%%%%%%%%%%%%%%%%%%
% HEADER
%%%%%%%%%%%%%%%%%%%%%%%%%%%%%%

\chead{
  {\vbox{
      Machine Learning \& Data Mining \hfill
      Caltech CS/CNS/EE 155 \hfill \\[1pt]
      Miniproject 1\hfill
      February 2025 \\
    }
  }
}

\begin{document}
\pagestyle{fancy}

%%%%%%%%%%%%%%%%%%%%%%%%%%%%%%
% PROBLEM 1
%%%%%%%%%%%%%%%%%%%%%%%%%%%%%%

\newpage

\section{Introduction [15 points]}
\begin{itemize}
  \item Group members: Chengrui Qu (), Chengyi Liu (), Mengzhan Liufu (), Natalia Zhang (2021011832)
  \item Kaggle team name: (Anonymous Team)
  \item Ranking on the private leaderboard:
  \item AUC score on the private leaderboard:
  \item Colab link:
  \item Piazza link:
  \item Division of labor:
\end{itemize}
\newpage

\section{Overview [15 points]}
% 1 page maximum
\subsection{Models and techniques attempted}
Fill in

\subsection{Work timeline}
Fill in

\newpage

\section{Approach [20 points]}
% 1 page maximum
\subsection{Data exploration, processing and manipulation}
Fill in

\subsection{Details of models and techniques}
Fill in

\newpage

\section{Model Selection [20 points]}
% 1 page maximum
\subsection{Scoring}
Fill in

\subsection{Validation and test}
Fill in

\newpage

\section{Conclusion [20 points]}
% 1 page maximum
\subsection{Insights}
Fill in

\subsection{Challenges}
Fill in
\newpage

\section{Extra Credit [5 points]}

\newpage

\section{LLM Usage}
This section has an example of LLM usage reporting below. Please follow this format to report LLM usage. Indicate each usage clearly.
\begin{itemize}
  \item Name of LLM(s) Used: ChatGPT
  \item Components of Project involving LLM: Data processing, Workflow management, Debugging.
\end{itemize}

\subsection{Data preprocessing}

\subsection{Debugging}
We used the LLM to assist with debugging our code by providing explanations of error messages and suggesting potential fixes.

\begin{figure}[htbp]
  \centering
  \includegraphics[width=0.68\textwidth]{../img/LLM_debug.png}
  \caption{Integration of LLM's output for debugging}
  \label{fig:llm-debug}
\end{figure}

\subsection{Workflow management}

As shown in Figure~\ref{fig:llm-workflow}, we integrated the LLM's output for workflow management by using it to generate a structured plan for our project. 
The LLM provided a step-by-step outline of the tasks we needed to complete, which helped us stay organized and on track throughout the project. 
We used the LLM's suggestions to prioritize our work and allocate time effectively, ensuring that we addressed all necessary components of the project in a timely manner.
\begin{figure}[htbp]
  \centering
  \includegraphics[width=0.68\textwidth]{../img/LLM_workflow.png}
  \caption{Integration of LLM's output for workflow management}
  \label{fig:llm-workflow}
\end{figure}

% \begin{figure}[t]
%   \centering
%   \includegraphics[width=0.68\textwidth]{figs/llm_report_2.png}
%   \caption{Integration of LLM's output for data preprocessing}
%   \label{fig:data-prep-integration}
% \end{figure}

\end{document}
